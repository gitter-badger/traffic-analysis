%!TEX root = ../dokumentation.tex

\chapter{Einleitung}
Heutzutage haben immer mehr Mitarbeiter innovative Ideen. Selten werden von einem Mitarbeiter konkrete Ideen oder Innovationen kund getan, dahingegen bleiben diese Ideen häufig unentdeckt. Mitarbeiter haben entweder das Gefühl, zur Kundgebung ihrer Ideen keine passende Möglichkeit zu besitzen oder trauen sich schlicht nicht, diese mitzuteilen. Um den Mitarbeitern eine transparente Plattform für ihre Ideen zu bieten, wird das interne Crowdfunding in Betracht gezogen. In den vergangenen Jahren hat sich Crowdfunding bereits als eine alternative Finanzierungsform immer beliebter gemacht \cite{tableofvisions}. Seit einigen Jahren wird von namhaften Unternehmen wie Audi, Daimler und IBM internes Crowdfunding dazu eingesetzt, Ideen und Innovationen über die eigene Mitarbeiter Crowd zu generieren \cites{innosabi, crowdfunding}. Im Zuge dieser Arbeit soll die Akzeptanz als auch Realisierbarkeit einer internen Crowdfunding Plattform bei der Firma camos Software und Beratung GmbH (im Folgenden als \emph{camos} bezeichnet) analysiert werden. Hierfür soll der bestehende Prozess der Ideen- und Innovations Publikation bei camos analysiert und sukzessive in einen Crowdfunding Prototypen überführt werden.

\section{Abgrenzung der Arbeit}
Das im Umfang dieser Arbeit aufgeführte Konzept wurde unter Berücksichtigung einer ausführlichen Umfrage aller Mitarbeiter der Firma camos entwickelt. Die Konzeption ist daher spezifisch auf das Unternehmen zugeschnitten und nicht zwangsläufig allgemeingültig anwendbar. 

\section{Einordnung in den Kontext der Arbeit}
Diese Arbeit behandelt eine konzeptionelle Analyse der Ideen- und Innovations Publikation der Firma camos, mit anschließender Konzeption eines Crowdfunding Prototypen. Für die Erstellung des Konzepts werden die damit verbundenen Literaturen genannt und bewährte Best-Practices bekannter Plattformen berücksichtigt.

\section{Spezifikation des Vorgehens}
Zu Beginn dieser Ausarbeitung werden in den folgenden Kapiteln zunächst die theoretischen Grundlagen erläutert. Anschließend werden der Ist- und Soll-Zustand definiert. Die darauf folgende Konzeption orientiert sich am Soll-Zustand unter Zuhilfenahme verschiedener Literaturen und Publikationen. Zu Beginn muss eine Erläuterung der technischen Begriffe und Aspekte, welche im Konzept verwendet werden, geschehen. Diese Erläuterung wird mit dem theoretischen Hintergrund im Folgenden Kapitel vorgenommen.